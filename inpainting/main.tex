% Copyright 2004 by Till Tantau <tantau@users.sourceforge.net>.
%
% In principle, this file can be redistributed and/or modified under
% the terms of the GNU Public License, version 2.
%
% However, this file is supposed to be a template to be modified
% for your own needs. For this reason, if you use this file as a
% template and not specifically distribute it as part of a another
% package/program, I grant the extra permission to freely copy and
% modify this file as you see fit and even to delete this copyright
% notice. 

\documentclass{beamer}

% There are many different themes available for Beamer. A comprehensive
% list with examples is given here:
% http://deic.uab.es/~iblanes/beamer_gallery/index_by_theme.html
% You can uncomment the themes below if you would like to use a different
% one:
%\usetheme{AnnArbor}
%\usetheme{Antibes}
%\usetheme{Bergen}
\usetheme{Berkeley}
%\usetheme{Berlin}
%\usetheme{Boadilla}
%\usetheme{boxes}
%\usetheme{CambridgeUS}
%\usetheme{Copenhagen}
%\usetheme{Darmstadt}
%\usetheme{default}
%\usetheme{Frankfurt}
%\usetheme{Goettingen}
%\usetheme{Hannover}
%\usetheme{Ilmenau}
%\usetheme{JuanLesPins}
%\usetheme{Luebeck}
%\usetheme{Madrid}
%\usetheme{Malmoe}
%\usetheme{Marburg}
%\usetheme{Montpellier}
%\usetheme{PaloAlto}
%\usetheme{Pittsburgh}
%\usetheme{Rochester}
%\usetheme{Singapore}
%\usetheme{Szeged}
%\usetheme{Warsaw}
\usepackage{amsmath,graphicx,subfigure}
\title{Get out of my Picture!}

% A subtitle is optional and this may be deleted
%\subtitle{Optional Subtitle}

\author{Sai Chowdary G }
% - Give the names in the same order as the appear in the paper.
% - Use the \inst{?} command only if the authors have different
%   affiliation.

\institute[IITGN] % (optional, but mostly needed)
{
  %
  Electrical Engineering\\
  Indian Institute of Technology, Gandhinagar
  }
% - Use the \inst command only if there are several affiliations.
% - Keep it simple, no one is interested in your street address.

\date{CS 603 - Computational Photography\\ Project Presentation}
% - Either use conference name or its abbreviation.
% - Not really informative to the audience, more for people (including
%   yourself) who are reading the slides online


% This is only inserted into the PDF information catalog. Can be left
% out. 

% If you have a file called "university-logo-filename.xxx", where xxx
% is a graphic format that can be processed by latex or pdflatex,
% resp., then you can add a logo as follows:

% \pgfdeclareimage[height=0.5cm]{university-logo}{university-logo-filename}
% \logo{\pgfuseimage{university-logo}}

% Delete this, if you do not want the table of contents to pop up at
% the beginning of each subsection:
\AtBeginSubsection[]
{
  \begin{frame}<beamer>{Outline}
    \tableofcontents[currentsection,currentsubsection]
  \end{frame}
}

% Let's get started
\begin{document}

\begin{frame}
  \titlepage
\end{frame}

\begin{frame}{Outline}
  \tableofcontents
  % You might wish to add the option [pausesections]
\end{frame}

% Section and subsections will appear in the presentation overview
% and table of contents.
\section{Introduction}

\subsection{Inpainting}

\begin{frame}{Inpainting}
  \begin{itemize}
  \item {
    Image inpainting generally is the process of filling up lost(or deliberately masked) regions of the scene.
  }
  \item {
  There are a wide variety of techniques that have been developed to address this issue
  }
 
  \end{itemize}
\end{frame}

\begin{frame}{A few Inpainting techniques}
  \begin{itemize}
  \item {
   Diffusion tehcniques based on PDE's \cite{1}
  }
  \item {
  Exemplar based image inpainting techniques \cite{2}
  }
  \item {
  Texture based image inpainting \cite{3}
  }
   \item {
  Structure based image inpainting \cite{4}
  }
   \item {
  But methods have the limitation that the size of the patch that need to be filled has to be small and also most of these use information from the some other part in the same image to fill the patch. These methods are ideal for removing small objects like wires etc. 
  }
  \end{itemize}
\end{frame}
\begin{frame}{A few Inpainting techniques}
\item {
  Inpainting using images of semantically similar scenes \cite{5}
  }
   \item {This technique uses information from other semantically similar scenes to fill the patch}
    \item {This does not give us the true scene}
\end{frame}

\begin{frame}{Our method}
\item {
  Inpainting using images of same scene taken from different views \cite{6}
  }
   \item {This technique uses information from other images of the same scene taken from a different view, to fill the patch}
    \item {The information occluded in one view might be visible from another view and thus can be used to fill the patch.}
\end{frame}
\section{Methodology}

% You can reveal the parts of a slide one at a time
% with the \pause command:
\begin{frame}{Image equation}

\end{frame}

\subsection{The problem}

\begin{frame}{Problem}
\begin{block}{Problem}
Given a blurred image $y$, and a filter $f$ our problem now is to find the sharp image $x$.
\end{block}
\pause
\begin{block}{Proposed Approach}
Some methods to solve for $x$ include Richardson-Lucy algorithm, deconvolution using a gaussian prior, and deconvolution using a spare prior.
\end{block}

\end{frame}

\subsection{Proposed Approach}
\begin{frame}{Proposed Approach}
  \begin{itemize}
  \item {
    If we consider convolution as a linear operator, the previous equation can be rewritten as,
\begin{equation}
y = C_fx
\end{equation} 
where $C_f$  is an $N \times N$ convolution matrix, the images $x,y$ are written as $N \times 1$ element vectors.
    
    \pause % The slide will pause after showing the first item
  }
  \item {   
    In frequency domain, the convolution formula is as follows. 
\begin{equation}
Y(v,\omega) = F(v,\omega)X(v,\omega)
\end{equation}

where $X, Y$ are frequency domain representations of $x, y$ and $v, w$ are their coordinates in frequency domain.
  }
  \end{itemize}
\end{frame}

\begin{frame}{Proposed Approach}
  \begin{itemize}
  \item {
   If $C_f$ is a full rank matrix and no noise is involved in imaging process, then we can get $x$ by simply applying the following equation.
\begin{equation}
x = C_f^-1y
\end{equation} 
    \pause % The slide will pause after showing the first item
  }
  \item {   
    And in frequency domain by the following equation.
\begin{equation}
X(v,\omega) = \frac{Y(v,\omega)}{F(v,\omega)}
\end{equation}
but the above equation won't be defined in the cases where $F(v,\omega) = 0$. 
  }
  \end{itemize}
\end{frame}

\begin{frame}{Proposed Approach}
  \begin{itemize}
  \item {
   Now, let us observe the case when $F(v,\omega)$ is not equal to $0$ but small. 
    \pause % The slide will pause after showing the first item
  }
  \item {   
    This will be a problem in those case when we have noise in imaging process.
\begin{eqnarray}
Y(v,\omega) = F(v,\omega)X(v,\omega) + n\\
\frac{Y(v,\omega)}{F(v,\omega)} = X(v,\omega) + \frac{n}{F(v,\omega)}
\end{eqnarray}
\pause
  }
  \item {
  	If we observe the above equation, the noise contribution will be increased when $F(v,\omega)$ is small.
  }
  \end{itemize}
\end{frame}

\begin{frame}{Proposed Approach}
  \begin{itemize}
  \item {
   To overcome these difficuties we will find the maximum a-posteriori explanation for x given y:
\begin{equation}
x = argmax_xP(x|y) \propto P(y|x)P(x)
\end{equation} 
    \pause % The slide will pause after showing the first item
  }
  \item {   
    If we assume that we have independent and identically distributed gaussian noise with variance $\eta$, then we can express the likelihood as follows,
\begin{equation}
P(y|x) \propto e^{-\frac{1}{2\eta^2} \parallel x-C_fy \parallel^2}
\end{equation}
and priori on x can be defined as
\begin{equation}
P(x) = e^{-\alpha\sum_{i,k}\rho(g_{i,k}*x)}
\end{equation}
where $g_{i,k}$ is the $k^{th}$ filter centered at pixel $i$ of the image and $i$ sums over image pixels. The fucntion $\rho$ is either sparse or heavy tailed function and in the paper they used 
\begin{equation}
\rho(z) = |z|^{0.8}
\end{equation}
  }
  \end{itemize}
\end{frame}

\begin{frame}{Proposed Approach}
  \begin{itemize}
  \item {
   The general selection of filters $g_k$ is the vertical and horizontal derivative filters namely $g_x = [1, -1]$ and $g_y = [1, -1]^T$. 
    \pause % The slide will pause after showing the first item
  }
  \item {   
    If we take the log of equations 8, 9 and 10, the maximum a-posteriori  explanation of x is given by
\begin{equation}
x = argmin_x \parallel y - C_fx \parallel^2 + W \sum_{i,k} \rho(g_{i,k}*x)
\end{equation}
where $W = \alpha\eta^2$ and $\parallel y - C_fx \parallel^2$ is called the reconstruction error.\pause
  }
\item {
	The above equation can be solved using various techniques out of which we will specifically focus on deconvolution using a gaussian prior and deconvolution using a sparse prior.
}  
  
  \end{itemize}
\end{frame}

\subsection{Deconvolution using a gaussian prior}
\begin{frame}{Deconvolution using a gaussian prior}
  \begin{itemize}
  \item {
   Differentiate the equation 12 with respect to $x$ and equate the derivative to zero. 
    \pause % The slide will pause after showing the first item
  }
  \item {   
    Now, we will get the optimal solution by solving the sparse set of linear equations.
\begin{equation}
Ax = b
\end{equation}
where $A$ and $b$ are given by
\begin{equation}
A = C_f^TC_f + W \sigma_k C_{g_k}^TC_{g_k}
\end{equation}
\begin{equation}
b = C_f^Ty
\end{equation}
  }
  
  
  \end{itemize}
\end{frame}

\begin{frame}{Deconvolution using a gaussian prior}
  \begin{itemize}
  \item {
   From equation 14, we can observe that $A$ in frequency domain is nothing but sum of convolution matrices and its diagonal. This can be shown by the following equation.
\begin{equation}
A_F = |F(v,\omega)|^2 + W \sum_k|G_k(v,\omega)|^2
\end{equation}
where $A_F$ is the representation of $A$ in frequency domain and similarly we can obtain $b_F$ as 
\begin{equation}
b_F = F(v,\omega)*Y(v,\omega)
\end{equation}
   % \pause % The slide will pause after showing the first item
  }
  
  \end{itemize}
\end{frame}

\begin{frame}{Deconvolution using a gaussian prior}
  \begin{itemize}
  \item {
   By using the equations 16 and 17, we can obtain $X$ as follows

\begin{equation}
X(v,\omega) = \frac{F(v,\omega)*Y(v,\omega)}{|F(v,\omega)|^2 + W \sum_k|G_k(v,\omega)|^2}
\end{equation}
where $W$ is the prior. \pause % The slide will pause after showing the first item
  }
  \item {
  	The prior effect will be larger at higher frequencies. At higher frequencies, the image content will be generally small and noise content will be large.
  }
  \item {
  	In gaussian image prior the optimization problem is convex and optimal solution can be obtained in closed form.
  }
  \end{itemize}
\end{frame}

\subsection{Deconvolution using a sparse prior}
\begin{frame}{Deconvolution using a sparse prior}
  \begin{itemize}
  \item {
   In natural images, the distribution of derivative filters won't be gaussian but sparse.
    % The slide will pause after showing the first item
  }
  \item {   
    A sparse prior concentrates derivatives at a smaller number of pixels when compared to gaussian prior.
  }
  \item {
  	(Advantages) This will leave majority of image pixels as constant which will produce \alert{sharper edges and noise reduction} with removal of unwanted artifacts such as ringing.
  }
  \item {
  	(Disadvantages) The optimization problem is \alert{not convex}, it can't be minimized in closed form. So, the optimization is achieved by iterative re-weighted least squares process (IRLS).
  }
  \end{itemize}
\end{frame}

\begin{frame}{IRLS Algorithm}
  \begin{itemize}
  \item {
   In the IRLS approach the cost minimization is approached in the following form.
\begin{equation}
\sum_j\rho(A_{j \to x} - b_j)
\end{equation}  
  }
  \item {   
    If we set the row vectors $A_{j \to}$ as the rows of the convolution matrices in equation 14, $C_{gk}$ and $C_f$, we will approach to deblurring problem.
  }
  \item {
  	In IRLS problem, each least square in the present step is re-weighted by its solution in the previous step. 
  }
  \end{itemize}
\end{frame}

\begin{frame}{IRLS Algorithm}
  \begin{itemize}
\item Initialize: $\psi_j^0 = 1$
\item Repeat till convergence:
\begin{itemize}
\item Let $A' = \sum_jA_{j\to}^T\psi_j^{t-1}A_{j\to}$ and\\ $b' = \sum_jA_{j\to}^T\psi_j^{t-1}b_j$. $x^t$ will be the solution for $A'x = b'$.
\item Set $u_j = A_{j\to}x^t-b_j$ and $\psi_j^t(u_j) = \frac{1}{u_j}\frac{d\rho(u_j)}{du}$ where \\
 $\rho (u_j) = |u_j|^{0.8}$ which implies the reweighting term\\ $\psi (u_j) = max(|u_j|,\epsilon)^{0.8-2}$
\end{itemize}
where $|u_j|$ is replaced by $max(|u_j|,\epsilon)$ to eliminate division by zero. 
\item The above algorithm can be implemented only in spatial domain because of re-weighting process. 
\item The quality of the results will be dependent on the number of iterations we are using for conjugate gradient.
  \end{itemize}
\end{frame}

\section{Results}
\begin{frame}{Lucy Deconvolution}
\begin{figure}
\centering
\subfigure[Original Image]{\includegraphics[width=0.4\textwidth]{images/original.png}}
\subfigure[Lucy Deconvolution]{\includegraphics[width=0.4\textwidth]{images/lucy.png}}
\label{fig:1}
\end{figure}
\end{frame}

\begin{frame}{Deconvolution using Gaussian Priors}
\begin{figure}
\centering
\subfigure[Gaussian prior, $w$ = 0.002]{\includegraphics[width=0.33\textwidth]{images/freq002.png}}
\subfigure[Gaussian prior, $w$ = 0.005]{\includegraphics[width=0.33\textwidth]{images/freq005.png}}
\subfigure[Gaussian prior, $w$ = 0.01]{\includegraphics[width=0.33\textwidth]{images/freq01.png}}
\subfigure[Gaussian prior, $w$ = 0.05]{\includegraphics[width=0.33\textwidth]{images/freq05.png}}
\label{fig:2}
\end{figure}
\end{frame}

\begin{frame}{Deconvolution using Sparse Priors}
\begin{figure}
\centering
\subfigure[Sparse prior, $w$ = 0.001]{\includegraphics[width=0.33\textwidth]{images/sps001.png}}
\subfigure[Sparse prior, $w$ = 0.004]{\includegraphics[width=0.33\textwidth]{images/sps004.png}}
\subfigure[Sparse prior, $w$ = 0.008]{\includegraphics[width=0.33\textwidth]{images/sps008.png}}
\subfigure[Sparse prior, $w$ = 0.01]{\includegraphics[width=0.33\textwidth]{images/sps01.png}}
\label{fig:3}
\end{figure}
\end{frame}

\begin{frame}{Applications}
\begin{figure}
\centering
\subfigure[Defocus Image]{\includegraphics[width=0.33\textwidth]{images/beer_coke_inp.png}}
\subfigure[All-focus Image]{\includegraphics[width=0.33\textwidth]{images/beer_coke_out.png}}
\subfigure[Defocus Image]{\includegraphics[width=0.33\textwidth]{images/cups_board_inp.png}}
\subfigure[All-focus Image]{\includegraphics[width=0.33\textwidth]{images/cups_board_out.png}}
\label{fig:4}
\end{figure}
\end{frame}

% Placing a * after \section means it will not show in the
% outline or table of contents.
\section*{Summary}

\begin{frame}{Summary}
  \begin{itemize}
  \item
    In this project, we have implemented the different deconvolution techniques proposed in the \cite{levin2007image}.
  \item
    We also discussed about the demerits and merits of different convolution techniques and how they are implemented.
  \item
    We also discussed about some applications of the proposed approach and how to proceed to those applications from the proposed algorithm.
  \end{itemize}
  
\end{frame}





% All of the following is optional and typically not needed. 
\appendix
\section<presentation>*{\appendixname}
\subsection<presentation>*{For Further Reading}

\begin{frame}[allowframebreaks]
  \frametitle<presentation>{Reference paper}
    
  \begin{thebibliography}{10}
    
  \beamertemplatebookbibitems
  % Start with overview books.

  \bibitem{1}
    Bertalmio, Marcelo and Sapiro, Guillermo and Caselles, Vincent and Ballester, Coloma
    \newblock {\em Image inpainting }.
    \newblock Proceedings of the 27th annual conference on Computer graphics and interactive techniques,2000.
    \bibitem{2}
   Criminisi, Antonio and P{\'e}rez, Patrick and Toyama, Kentaro
    \newblock {\em Region filling and object removal by exemplar-based image inpainting }.
    \newblock Image Processing, IEEE Transactions on,2004.
    \bibitem{3}
  Efros, Alexei A and Leung, Thomas K
    \newblock {\em Texture synthesis by non-parametric sampling }.
    \newblock Computer Vision, 1999. The Proceedings of the Seventh IEEE International Conference on
 \bibitem{4}
  Sun, Jian and Yuan, Lu and Jia, Jiaya and Shum, Heung-Yeung
    \newblock {\em Image completion with structure propagation }.
    \newblock ACM Transactions on Graphics (ToG),2005.
\bibitem{5}
  Hays, James and Efros, Alexei A
    \newblock {\em Scene completion using millions of photographs }.
    \newblock ACM Transactions on Graphics (ToG),2007.
    \bibitem{6}
   Whyte, Oliver and Sivic, Josef and Zisserman, Andrew
    \newblock {\em Get Out of my Picture! Internet-based Inpainting. }.
    \newblock BMVC,2009.

  \end{thebibliography}
\end{frame}

\end{document}


